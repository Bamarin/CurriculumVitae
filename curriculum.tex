% Righe di impostazioni per TeXworks e TeXstudio
% !TEX encoding = UTF-8
% !TEX program = pdflatex
% !TEX spellcheck = it_IT

\documentclass[11pt,a4paper,sans]{moderncv}

\moderncvstyle{casual}
\moderncvcolor{nicolas}

% alcuni pacchetti standard
\usepackage[english,italian]{babel} % solo se si scrive in italiano
\usepackage[utf8]{inputenx}
\usepackage[left=2cm,right=2cm,top=1.8cm,bottom=2.2cm]{geometry}
%\usepackage{fontspec}

\AfterPreamble{
	\hypersetup{
		colorlinks=true,
		linkcolor=color1,
		urlcolor=color1
	}
}
% questa riga allarga la colonna di sinistra
\setlength{\hintscolumnwidth}{3.7cm}

% personal data
\firstname{Nicolas}
\familyname{Farabegoli}
\title{Curriculum Vitae}
\address{via Bagalona 599}{47032, Bertinoro (FC), Italy}
\mobile{+39\,340 2876022}
\email{nicolas.farabegoli@gmail.com}

\definecolor{color3}{HTML}{333333}
%\photo[70pt]{name.jpg}
\quote{\color{color3}"Qualcosa di figo e accativante qui"}

\begin{document}

\vspace*{-2cm}
\maketitle
\vspace*{-1cm}

\section{Informazioni personali}
\cvline{Nome}{Nicolas}
\cvline{Cognome}{Farabegoli}
\cvline{Luogo e data di nascita}{Forlì (FC), 20 Ottobre 1997}
\cvline{Nazionalità}{Italiana}

\section{Istruzione e Formazione}

\cventry{Set. 2016 -- presente}{Laurea triennale in Ingegneria e Scienze informatiche.\newline
Università di Bologna, campus di Cesena}{}{Cesena (FC)}{}{Cia a tutti}
\cvitem{Voto finale}{pending...}

\cventry{Set. 2011 -- Lug. 2016}{Diploma in "Elettrotecnica ed Elettronica" articolazione Elettronica.\newline Istituto Tecnico Tecnologico B. Pascal}{}{Cesena (FC)}{}{Durante il percorso triennale si sono apprese le principali conoscienze riguardo l'elettronica digitale ed analogica e loro applicazioni. Sono stati realizzati semplici impianti civili, industriali ad alta potenza comandati da PLC.  }
\cvitem{Voto finale}{90/100}

\section{Competenze linguistiche}
\cvline{\textbf{Italiano}}{Lingua madre}
\cvline{\textbf{Inglese}}{Certificazione linguistica B1}
\cvline{\textbf{Spagnolo}}{Livello scolastico, orale e scritto}

\section{Competenze tecniche}

\cvitem{Sistemi operativi}{Linux, MacOs, Microsoft Windows, iOS, Android}
\cvitem{Linguaggi di programmazione}{C/C++, C\#, Java, Java 8, Java Android, Bash, Python, ASM (IA-32)}
\cvitem{Linguaggi di Markup}{Markdown, RST, LaTeX}
\cvitem{IDE}{Visual Studio, Eclipse, Android Studio, AutoCAD, PSoCreator, Arduino IDE, Mplab X, TIA Portal, Piattaforma mbed, Atmel Studio, Keil}
\cvitem{Office Suite}{LibreOffice, Microsoft Office}
\cvitem{Editor grafici}{Photoshop, Adobe Illustrator, GIMP}
\cvitem{Browser}{Firefox, Chrome, Chromium, Safari}
\cvitem{Altro}{Interesse nel mondo linux e ARM, oltre al mondo della microelettronica per scopi general purpose.}
\end{document}

